%!TEX root = ../thesis.tex
\chapter*{Abstract}
\addcontentsline{toc}{chapter}{Abstract}


The three dimensional ultrasound computer tomography (USCT) is a novel imaging technique for the detection of early stage breast cancer in women. The combination of the basic principle of ultrasonic imaging with the computed tomography (CT) allows the reconstruction of high resolution three dimensional images of the examined breast tissue. Currently there are three modalities in use: reflection imaging, attenuation- and  speed-of-sound-transmission-imaging. For the reflection image a 3D synthetic aperture focusing technique (SAFT) is applied to calculated a reflectivity value for each voxel of the image. In this thesis new imaging techniques are proposed be complemented by a fourth modality which allows for the preservation of directional information of reflections. With the analysis of the reflection characteristics of the tissue a distinction tissue types with a high specular reflectivity and diffuse reflecting tissue types is made available. 

This directional information is gathered in additional dimensions by analysing and splitting the reflectivity information for incoming and outgoing direction of the ultrasound wave for each voxel.

Until now onlys 4D information.

This thesis 5D Information

In previous works the imaging volume was extended from 3D to 4D to either save the emitter information of the voxel or the receiver information. With the introduction of a 5th dimension it is possible to save both directional information and allows the assertion of the reflection characteristic of the sample which was not possible in 4D before. A generalisation of the discretisation of incoming and outgoing direction is introduced which allows to arbitrarily increase the resolution of the segmentation. Furthermore, a more efficient method for the assignment of vectors to the discretised directional segmentation is presented which decreases the execution time by multiple orders. This thesis provides the tools for detailed classification of tissue types for the early detection of breast cancer. 

