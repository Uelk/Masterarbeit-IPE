%!TEX root = ../thesis.tex
\chapter*{Abstract}
\addcontentsline{toc}{chapter}{Abstract}


The three dimensional ultrasound computer tomography (USCT) is a novel screening technique for the detection of early stage breast cancer in woman. The combination of the basic principle of ultrasonic imaging with the computer tomography (CT) allows the reconstruction of high resolution three dimensional images of the examined breast tissue. Currently there are three modalities in use: reflection imaging, attenuation- and  speed-of-sound-transmission-imaging. For the reflection image a 3D synthetic
aperture focusing technique (SAFT) is applied to calculated a reflectivity value for each voxel of the image. The speed-of-sound-map and attenuation image are yielded by an algebraic reconstruction technique. These imaging techniques shall be complemented by a fourth modality which allows for the preservation of directional information. With the analysis of the reflection characteristics of the tissue samples a distinction between homogeneous and even tissue types with a high specular reflectivity and diffuse reflecting tissue types is made available. The current implementation of the reconstruction algorithm does not regard the direction during the calculation of the image. 
New approaches for the conservation of directional information are presented as well as existing ones are extended. In previous works the imaging volume was extended from 3D to 4D to either save the emitter information of the voxel or the receiver information. With the introduction of a 5th dimension it is possible to save both directional information and allows the assertion of the reflection characteristic of the sample which was not possible in 4D before. The discretisation of the directional information of each image voxel is a big part of this thesis. A generalisation of the problem is introduced which allows to arbitrarily increase the resolution of the segmentation. Furthermore, a more efficient method for the assignment of vectors to the discretised directional segmentation is presented which decreases the execution time by multiple orders. This thesis provides the tools for detailed classification of tissue types for the breast cancer early detection. 

