%!TEX root = ../thesis.tex
\chapter*{Abstract}
\addcontentsline{toc}{chapter}{Abstract}


The three dimensional ultrasound computer tomography (USCT) is a novel imaging technique for the detection of early stage breast cancer in women. The combination of the basic principle of ultrasonic imaging with the computed tomography (CT) allows the reconstruction of high resolution three dimensional images of the examined breast tissue. Currently there are three modalities in use: reflection imaging, attenuation imaging and speed of sound transmission imaging. For the reflection image a 3D synthetic aperture focusing technique (SAFT) is applied to calculate a reflectivity value for each voxel of the image.

In this thesis new imaging techniques are proposed which allow for the preservation of directional information of reflections. With the analysis of the reflection characteristics of the tissue a distinction of tissue types with a specular reflectivity or diffuse reflecting tissue types is aimed for. 


In a previous work a prototypical implementation of a method for the analysis of the directional information was conducted, in which only a four dimensional image had been captured. This thesis extends the extraction of directional information into the five dimensional space by considering incoming and outgoing directions separately.


A generalisation of the discretisation problem of incoming and outgoing directions is introduced which allows to find a trade-off between the memory consumption and resolution of the angular segmentation. Furthermore, a more efficient method for the assignment of vectors to the discretised directional segmentation is presented which can decrease the execution time by multiple orders. The developed methods have been fully integrated into the clinically used image reconstruction software, which also allows to perform a sound speed correction to increase the resolution and contrast of the image.

The methods have been evaluated with experimental data. The process of assigning the reflectivity information to individual incoming and outgoing directions has been successfully verified. An analysis of the reflection characteristics for distinct material types show that these materials can be distinguished. 
For the three dimensional visualisation of the five dimensional results different methods are presented. Furthermore, it is shown that the sound speed correction makes the differentiation of material types more distinct.

This thesis lays the foundation for in-depth analysis of reflection characteristics of different breast tissue and may provide the tools for the detailed classification of tissue types for the early detection of breast cancer.