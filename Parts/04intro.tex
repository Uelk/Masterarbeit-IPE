%!TEX root = ../thesis.tex
\chapter{Introduction}
\label{chap:introduction}


The last couple of decades have shown a steady increase of the incidence for female breast cancer.
% An US based study published in March 2020 \cite{Lima2020Trends2015} has shown that the cancer incidence for women between the age of 25 to 39 has increased by 0.65\% per year.
In 1935 the incidence for breast cancer was at 16.3 diagnoses per 100.000 women whereas 38.5 per 100.000 women were diagnosed in the year 2015 \cite{Lima2020Trends2015}.
In Germany 71.600 women were diagnosed with breast cancer in the year 2013 with cases having doubled since 1970. A more fortunate development has been observed in the death rate in patients diagnosed with breast cancer. Since 1999 the mortality rate decreased by about a third in patients under 50 and was 25\% lower in patients between the age 50-69 during that period. 
% Some of the various reasons for the increased incidence are presumably the higher age at which women become pregnant for the first time, hormone based birth control as well as a change in diet and in exercise \cite{RobertKoch-Institut2016Bericht2016}.
One of the reasons that the mortality rate could be decreased is the introduction of the screening program for women between the ages 50-69 \cite{Hubner2020Long-termGermany}. Since then the number of diagnoses of advanced tumour stages were decreased \cite{RobertKoch-Institut2016Bericht2016}.
The prognosis for the treatment of cancer depends on the stage of the cancer and how early it was discovered, as with the size of the tumour the risk of spreading into other organs increases. As the mortality rate directly correlates with the tumour size and stage of the cancer, an early detection of cancerogenous tissue is one of the most effective ways to increase the survival rate of patients \cite{Veronesi1985PrognosisNodes}, \cite{Welch2016Breast-CancerEffectiveness}.

% Among others some of the first symptoms of breast cancer often are changes in size or form of the breast as well as the formation of lumps within the breast  \cite{NationalInstitutesofHealthNIH-NationalCancerInstituteNCIBreastTreatment}.

The female breast consists of mainly fatty tissue and the mammary glands also known as lobules which can produce milk. Additionally, the ducts (see Figure \ref{anatomy_breast}) are part of the breast tissue which form the transport channels for the milk between the lobules and the nipple.

%A very common type of breast cancer arises from this ductal system of the breast. The first symptom known as Ductal Carcinoma are calcification i.e. the deposition of calcium in the tissue of these glands.
%This is a prestage of breast cancer known as stage 0 and is detectable by early mammography screening \cite{brestcancer_stages}.  


\begin{figure}[H]
    \centering
    \includegraphics[width=0.65\textwidth]{Graphics/breast.jpg}
    \caption{Schematic of the female breast. The lobes form the milk producing system of the breast whereas the ducts are channels for the milk towards the nipple of the breast. Source: \cite{mayo-clinic}. }
    \label{anatomy_breast}
\end{figure}

\hspace{-1cm}
   
The new imaging technique \ac{usct} uses the basic principles of the ultrasound imaging technique which are extended by some principle elements of the the \ac{ct}. For the measurement an ultrasound transducer emits the ultrasound waves into the imaging aperture of the \ac{usct}-system. The receivers record the pressure over time and yield a so-called \ac{ascan}. With the combination of the \acp{ascan} yielded from all emitter-receiver-combinations, a reflection and transmission images of the examined tissue can be reconstructed. The combination of multiple modalities like the reflection and speed of sound for the imaging can be used to generate high resolution images.

Compared to the widely used screening method, the X-ray mammography, which uses ionizing radiation, the three dimensional \ac{usct} has advantages and makes this novel imaging technique a promising alternative: By utilizing the ultrasound, high resolution 3D images of the breast tissue can be yielded without the use of ionizing radiation. Furthermore, the examination process with an \ac{usct} is much more comfortable for the patient since the breast does not have to be deformed during the imaging procedure. 





\section{Motivation and objectives}
\label{chap:motivation}

%The characterisation of reflections of different tissue types in one major objective of this thesis.

Until now reflectivity imaging calculates the qualitative average of the reflectivity of the tissue, however the scattering characteristics of the tissue can not be characterised from these data. It is assumed that tumour tissue consists of an inhomogeneous distribution of cancer cells which results in a distinct backscattering behaviour of this particular tissue. Similar to optical approaches it can further be assumed that small particles in the tissue have a omnidirectional reflection characteristic like a point scatterer. Tissue types with a predominant amount of inhomogeneously distributed small particles are therefore assumed to have diffuse reflection properties whereas larger, even tissue patches with homogeneous structure have a more specular reflection characteristic.
By introducing the methods for evaluating the reflection characteristic in the reconstruction algorithm, the reflection properties of different tissue types can be analysed.


Previous approaches resulted in a four dimensional image. The limitations of the four dimensional approach is that the actual back scattering characteristic can not be assessed without a 2\textsuperscript{nd} directional information. The 4\textsuperscript{th} dimension holds the information of the voxel-receiver configuration. With the 5\textsuperscript{th} dimension also the voxel-emitter relation can be regarded and used to analyse the reflection characteristics of the tissue. These limitations were tackled by extending the algorithm of the previous implementation into a fifth dimension. Thus, not only the scalar information about the outgoing ultrasound wave from the voxel to the receiver is regarded but also the direction of the incoming pulse from the emitter to the voxel. The directional information between the fourth and fifth dimension will also allow for the analysis of reflection properties of the tissue concerning its specular and diffuse reflection parts. This allows the determination of a specific reflection characteristic of the tissue.

The usage of platonic solids for the generation of directional vectors to discretise the directional information of the voxel in the work of Patrick Hucker \cite{PatrickHucker2014EvaluationRuckstreumodells} limited the resolution of the directional vectors and with that the capabilities to distinguish different reflection types. The generalisation of the discretisation problem of the measurement volume from small platonic solids with a limited number of directional vectors to the generation of arbitrary number of directional vectors with a higher resolution of the discretisation will increases the capabilities of the reflection analysis further. A simple example for different reflection types is given in Figure
\ref{relfec_chara_examp}:

\begin{figure}[H]
    \centering
    \includegraphics[width=0.56\textwidth]{Graphics/different_reflec_gauss.jpg}
    \caption{Example of three different types of reflection characteristics. The green bell curve shows close to specular reflectivity whereas the orange and yellow curves have a more diffuse reflection characteristic. }
    \label{relfec_chara_examp}
\end{figure}

To distinguish between the different reflection types it is necessary to reach a high enough discretisation of the direction information. In this thesis a new approach for the arbitrary extension of the resolution of the discretisation of the directional information is proposed, which overcomes the aforementioned limitations.

\medskip



Previous implementations for the analysis of the reflection characteristic lacked the capabilities of regarding the different \ac{sos} of tissues during the reconstruction. At that time a constant sound speed was assumed. Since then the calculation of the \ac{sos} for the different areas of the measurement aperture was introduced and the quality of the final image could be increased. In this work a \ac{sos}-correction will be introduced to the reconstruction algorithm and its influence on the resulting reflection characteristics compared to non-\ac{sos}-corrected results will be assessed.
Furthermore, the only prototypical implementation of the previous work as well as the new insights of this thesis were implemented in the clinically used reconstruction algorithm of the \ac{usct}-project with regard to downward-compatibility and data structure.




