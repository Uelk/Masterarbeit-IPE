%!TEX root = ../thesis.tex
\chapter{Introduction}
\label{chap:introduction}




The last couple of decades have shown a steady increase of the incidence for female breast cancer.
An US based study published in March 2020 \cite{Lima2020Trends2015} has shown that the cancer incidence for women between the age of 25 to 39 has increased by 0.65\% per year. In 1935 the incidence for breast cancer was at 16.3 diagnoses per 100.000 women whereas 38.5 per 100.000 women were diagnosed in the year 2015.
The steady rise of cases during the last 30 years has been observed in Germany in a similar manner as it has in America.

Compared to the widely used X-ray-based screening method which uses ionizing radiation the three dimensional \ac{usct} has its advantages and makes this novel imaging technique a promising alternative to classical mammography. By utilizing the ultrasound echo technique high resolution 3D image of the breast tissue can be yielded without the use of ionizing radiation. Furthermore, the examination process with an \ac{usct} is much more comfortable for the patient since the breast does not have to be deformed during the imaging procedure. 
During the scan the breasts are immersed in the aperture which is filled with water as coupling medium for the ultrasound measurement. The aperture is formed as a half sphere which is lined with over 2000 piezo-electric transducer arrays which are called \ac{tas}. During the measurement every for every emitter-receiver-combination an \ac{ascan} is recorded. 
From these data it is possible to reconstruct a picture of the breast tissue. For the reconstruction of the image the \ac{saft} is used.
Three modalities for the image reconstruction are currently in use: reflectivity, speed of sound and attenuation \cite{Jirik2012Sound-speedTomography}.
With these three modalities it is possible to reconstruct an image with a submillimetre-resolution.
The goal of this thesis was to introduce a fourth modality to the reconstruction algorithm to further increase the resolution of the image as well as to classify different tissue types by analysing the back scattering.  





\\

