%!TEX root = ./thesis.tex
% For details, see https://www.ctan.org/pkg/acro



%======== ADJUST SPACING BETWEEN ACRONYM AND FULL NAME HERE ===============================
% The following part adjusts the appearance of the printed list of abbreviations
% Based on: http://tex.stackexchange.com/questions/253338/abbreviation-list-align-left-with-space-between
% Generate your list so every thing is in line
\usepackage{enumitem}
\newlength\myitemwidth

\setlength\myitemwidth{9em} 						% <<< choose what you need here
\newlist{myacronymlist}{description}{1}
\setlist[myacronymlist]{
  labelindent = 0pt ,
  labelsep    = 0pt ,
  leftmargin  = \myitemwidth ,
  labelwidth  = \myitemwidth ,
  itemindent  = 0pt ,
  format      = \bfseries
}
% Tell acro to use that list					% Bugfix sp952 (23.01.2019)
\DeclareAcroListStyle{myliststyle}{list}{
  list = myacronymlist
}
\acsetup{ list-style = myliststyle } 			% Bugfix sp952 (23.01.2019)
%\acsetup{ list-type = myacronymlist } % comment for Bugfix l3sort 17.02.17 Wachter 
%=========================================================================================


%======== ENTER YOUR ABBREVIATIONS HERE ===============================

  % You can access the abbreviations in your code by the commands
  %  \ac for singluar
  %  \acp for plural
  %  \Ac for capitalized singular
  %  ... see acro package description for many more: https://www.ctan.org/pkg/acro

  % There is no need to sort your acronyms alphabetically, however it helps to have an overview.
  % You could assign classes for each acrynom, allowing to print only lists for eg. all channel types and a second list for all model types and so on.
  % This package acro is similar to the acronym package, but the latter does not provide capitalization

% short - short version of the acronym
% long - long version of the acronym
% long-plural-form - non-standard form of the pluar version
% list - abbreviation in the acronym list

  \DeclareAcronym{ACL}		{short={ACL},		long={atrial (basic) cycle length}}
  \DeclareAcronym{AFib}		{short={AFib},		long={atrial fibrillation}}
  \DeclareAcronym{AFlut}		{short={AFlut},		long={atrial flutter}}
  \DeclareAcronym{AP}		{short={AP},			long={action potential}}
  \DeclareAcronym{FT}		{short={FT},			long={atrial flutter}}
  \DeclareAcronym{FFT}		{short={FFT},		long={Fourier transform}}
  \DeclareAcronym{HC-PLSR}	{short={HC-PLSR},	long={hierarchical cluster-based partial least squares regression}}
  \DeclareAcronym{PiCA}		{short={$\pi$CA},	long={periodic component analysis}}
